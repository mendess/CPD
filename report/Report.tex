\documentclass[a4paper]{article}
\usepackage[utf8]{inputenc}
\usepackage[T1]{fontenc}
\usepackage[cache=false]{minted}
\usepackage[dvipsnames]{xcolor}
\usepackage{a4wide,syntax,listings,appendix,tikz,wrapfig,graphicx,hyperref}
\hypersetup{pdftitle={Processing an Angolan Newspaper},
pdfauthor={Pedro Mendes},
colorlinks=true,
urlcolor=blue,
linkcolor=black}
%\usetikzlibrary{arrows,positioning,automata,decorations.markings,shadows,shapes,calc}

\begin{document}

\title{TITLE HERE} %TODO:
\author{Pedro Mendes (ist197144)}
\date{\today}
\maketitle
%\tableofcontents

\section{Data Structure}
There was multiple approaches to store the matrix A, the most basic one was storing the values in a adjacency matrix however as the matrix A is a sparse one we opted to store it in a compressed sparse row (CSR). The CSR allows to store just the non-zero values in a single array making this a good choice in terms of caching since all the values are stored in adjancent memory positions

\section{Matrix B multiplication}
The new matrix B is calculated in every iteration however since the only positions that change are the ones who are different from zero in Matrix A we only multiply specific positions instead of the full matrix.
\section{Serial Vers}

\section{Problem}

\section{Solution}

\end{document}

